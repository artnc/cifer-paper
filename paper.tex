% For more information, see /templates/bare_conf.tex.

\documentclass[conference, draftcls]{IEEEtran}
\usepackage{cite}
\usepackage[pdftex]{graphicx}
\graphicspath{{../pdf/}{../jpeg/}}
\DeclareGraphicsExtensions{.pdf,.jpeg,.png}
\usepackage[cmex10]{amsmath}
\usepackage{algorithmic}
\usepackage{array}
\usepackage{mdwmath}
\usepackage{mdwtab}
\usepackage{eqparbox}
\usepackage{url}
\usepackage[tight,footnotesize]{subfigure}
\usepackage[caption=false,font=footnotesize]{subfig}
\usepackage{fixltx2e}
\usepackage{stfloats}

% correct bad hyphenation here
\hyphenation{op-tical net-works semi-conduc-tor}

\begin{document}
\title{CIFEr Tournament}

\author{
  \IEEEauthorblockN{Nachapon Arthur Chaidarun}
  \IEEEauthorblockA{
    College of Arts and Sciences\\
    University of Virginia\\
    Charlottesville, Virginia 22904\\
    Email: chaidarun@virginia.edu
  }
  \and
  \IEEEauthorblockN{Scott Tepsuporn}
  \IEEEauthorblockA{
    School of Engineering and \\ Applied Science \\
    University of Virginia\\
    Charlottesville, Virginia 22904\\
    Email: spt9np@virginia.edu
  }
}

% TODO Are we a special paper?
% use for special paper notices
%\IEEEspecialpapernotice{(Invited Paper)}

% make the title area
\maketitle

\begin{abstract}
%\boldmath
The abstract goes here.
\end{abstract}

\IEEEpeerreviewmaketitle

\section{Introduction}
Today's volatile markets have made hedge funds an attractive investment option. While hedge funds may underperform in the market at times \cite{amin2003}, the industry as a whole has grown from \$100 billion to \$2.5 trillon in the span of 16 years \cite{growth}. This growth has increased competition as well as the need for better hedging methods.

Like many strategies used in the financial industry, hedging techniques remain trade secrets. However, unlike simple stock trading, running a hedge fund requires a great deal more capital and therefore incurs a greater risk of failure. The fact that there does not yet exist a risk-free environment for practicing is worrying. Our hedge fund tournament hosted in conjunction with the 2014 IEEE CIFEr conference attempts to solve this problem.

Many solutions exist to provide a way for traders to practice stock trading; however the goal of these simulations such as \cite{wallstreetsurvivor} is normally to make the most money. This does not turn out to be a sufficient metric for hedge fund performance since a degree of luck may have been involved. What is needed instead is a way to measure risk and returns simultaneously. Our tournament will take this into account and provide a way for a wider audience to try their hand at managing a hedge fund. Furthermore, it will enable users to try different techniques and possibly publish their findings.

\section{Hedge Fund Tournament}
Describes the tournament.

\subsection{Player Rules}
Rules and algorithms.

\subsection{Implementation}
How the website functions.

\section{Price Series}
The price series in the hedge tournament are based on an historical S&P 500 price series. The first step in estimating the price series is to estimate the Beta of each equity (stock) asset, the benchmark being the S&P 500. Beta measures the relation between S&P500 returns and that of a specific stock. Using the Beta a price series can be generated that closely follows the real price series. To increase the difficulty of the tournament a U(-2\%,2\%) was added to generate a layer of noise. The noise adds a layer of randomness to the price series, as well as obscuring the historical time period.

Generating the option prices seen in the tournament require several additional steps. Firstly, the historical volatility for each stock was found. Volatility was measured as the standard deviation of log returns.  Additionally, an expiration date of the options needed to be selected. To limit the difficulty of the tournament the expiration date was chosen to outside the competition time window. This means teams did not have to worry about exercising the options. Furthermore, stocks were assumed not to pay a dividend, as this increase the difficulty of constructing an appropriate portfolio. It should be noted, that during the tournament time real world stocks did not pay dividends and therefore the authors’ contend this assumption is acceptable. The last assumptions the authors’ made was all options are European styled options. This means that the options cannot be exercised prior to the expiration date. Again this was to limit the difficulty in the tournament, allowing it to be accessible to people of different skill levels.

Using the simulated price series of each stock and the Black-Scholes formula a simulated option price series was generated. Similar to the stock price series a U(-2\%,2\%) was added to generate a layer of noise. Five puts and five call options were calculated for each stock. All options are originally created in the money. However, as the tournament progress the option may become worthless. The strike price of the options increment by \$0.50 or \$1.00, depending on if the underlying stock value started below or above \$15 respectively. The next section describes briefly several strategies that can be used to hedge such a portfolio.

\section{Hedging Strategies}

\section{Results}
Depends on the number of participants and the deadline of this paper...

\section{Conclusion}
Summarize and offer suggestions for future work.

\section*{Acknowledgment}
The authors would like to thank...

\bibliographystyle{IEEEtran}
\bibliography{./resources/bibliography}

\end{document}
